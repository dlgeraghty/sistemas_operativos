\documentclass[12pt, letterpaper]{article}
\usepackage[utf8]{inputenc}
\usepackage{fancyhdr}
\usepackage{listings}
\usepackage{xcolor}
\usepackage{hyperref}

\definecolor{codegreen}{rgb}{0,0.6,0}
\definecolor{codegray}{rgb}{0.5,0.5,0.5}
\definecolor{codepurple}{rgb}{0.58,0,0.82}
\definecolor{backcolour}{rgb}{0.95,0.95,0.92}

\hypersetup{
    colorlinks=true,
    linkcolor=blue,
    filecolor=magenta,
    urlcolor=cyan,
    pdftitle={Sharelatex Example},
    bookmarks=true,
    pdfpagemode=FullScreen,
}

\lstdefinestyle{mystyle}{
    backgroundcolor=\color{backcolour},
    commentstyle=\color{codegreen},
    keywordstyle=\color{magenta},
    numberstyle=\tiny\color{codegray},
    stringstyle=\color{codepurple},
    basicstyle=\ttfamily\footnotesize,
    breakatwhitespace=false,
    breaklines=true,
    captionpos=b,
    keepspaces=true,
    numbers=left,
    numbersep=5pt,
    showspaces=false,
    showstringspaces=false,
    showtabs=false,
    tabsize=2
}

\lstset{style=mystyle}

\title{Sistemas Operativos}
\author{David Lopez}
\date

\pagestyle{fancy}
\fancyhf{}
\rhead{David Lopez}
\lhead{Sistemas operativos}
\rfoot{\thepage}

\begin{document}

\begin{titlepage}
	\clearpage\maketitle
	\thispagestyle{empty}
\end{titlepage}

\tableofcontents{}

\section{conceptos basicos}

Un sistema operativo no es mas que un progama.

\subsection{Funciones del s.o}
Este programa, no obstante, es bastante caracteristico, pues se encarga de \textbf{gestionar los recursos hardware}, ofrece a los demas programas una api de \textbf{llamadas al sistema} y proporciona \textbf{mecanismos de interaccion con el usuario} como el shell.\par

Es necesario gestionar los recursos hardware por que podemos tener varios programas ejecutandose a la vez, y se hace mas importante con los sistemas multiusuario. Los programas competiran por los recursos, por lo que sera necesario administrarlos correctamente.\par

Sobre las llamadas al sistema habra algunos ejercicios para familiarizarnos, hay que tener en cuenta, que a diferencia de un curso de desarrollo sobre el kernel de un sistema operativo, aqui nos centraremos en programas de espacio de usuario y nos limitaremos, al menos por el momento a invocar a estas llamadas, no a implementarlas en incluirlas en el s.o , cosa que tambien se puede hacer.

\subsection{partes principales del s.o}

Podemos distinguir tres partes principales: \textbf{nucleo o kernel, servicios e interfaz de usuario}

\subsection{clasificacion segun su estructura}

\begin{itemize}
	\item{Monolitico (como linux): Tiene todos los componentes integrados en un mismo programa}
	\item{Por capas: cada capa ofrece una interfaz a la capa siguiente}
	\item{Cliente-Servidor (microkernel): el kernel se reduce a la minima expresion y el resto de utilidades se desarrollan en proceso de usuario}
\end{itemize}

\subsection{Como lo usamos ?}

Podemos poner el sistema operativo a trabajar, deliberadamente y para que haga la funcion que queremos, escribiendo un programa que realize llamadas al sistema, las interrupciones de los perifericos y los errores tambien provocara la activacion del s.o

\subsection{ejemplo de programa que invoca llamadas al sistema}
\textit{escribir una funcion en C, que actue como el comando cat de unix, es decir, que imprima el contenido del archivo por la salida estandar}

\lstinputlisting[language=C]{cat.c}

En este codigo, las funciones \textit{open, read, write y close} son llamadas al sistema, cuyo man se puede encontrar en: \href{http://man7.org/linux/man-pages/man2/}{manual de llamadas al sistema}, es fundamental saber utilizar esta herramienta y manejar con solvencia las llamadas al sistema mas habituales.

\section{Archivos y directorios}

\subsection{Conceptos basicos de archivos}
El sistema operativo nos oculta los detalles de la implementacion fisica del disco, proporcionandonos lo que conocemos como archivo.\par

\textit{Que es un archivo: Es una unidad de almacenamiento logico \textbf{no volatil} que agrupa un conjunto de informacion relacionada entre si bajo un mismo nombre}\par

Carateristicas de un archivo:

\begin{itemize}
	\item Desde el punto de vista de el sistema operativo este debe proporcionar al menos una estructura de archivo que permita realizar sobre el: darle nombre, proteccion y sistema eficiente de gestion de estos mismos. Ademas debe permitir anadirle los atributos que se deseen.
	\item El nombre y extension son basicos para un archivo y deben permitir que se le distinga inequivocamente
	\item El usuario debe percibir sencillez para manipular archivos. La traduccion entre las direcciones fisicas y logicas es lo que depende segun el sistema operativo pero se basa en el \textbf{mapa de archivos} que se almacena como atributo.
	\item Se distinguen dos tipos principales de acceso a la informacion en los archivos: \textbf{Secuencial}, si todos los bytes estan seguidos y en orden o \textbf{acceso aleatorio o directo} que permiten el acceso por conjuntos y desordenado.
	\item Cuando se accede de forma concurrente al mismo archivo, se establecen mecanismos tipicos para tratar la concurrencia: Archivos inmutables, sesiones(copias y versiones) y UNIX(cada proceso trabaja con su imagen independiente del archivo y no se permite que compartan el apuntador.)
\end{itemize}

\subsection{Conceptos basicos de directorios}
Para poder acceder a los archivos con facilidad, se crean formas de organizar los nombres de los archivos, son los directorios.

\textit{Que es un directorio: Es un objeto que relaciona el nombre del usuario de un archivo y el descriptor interno del mismo.}

Caracteristicas de los directorios:

\begin{itemize}
	\item Actualmente los sistemas operativos se organizan en forma de arbol, que representa los directorios y subdirectorios partiendo de una raiz de forma que existe un unico camino a un archivo.
	\item Para solventar la problematica de que varios usuarios quieran trabajar sobre el mismo archivo (pero tienen rutas hasta el distintas), se generaliza de arbol a grafo aciclico.
	\item La forma mas habitual de compartir archivos es crear un \textbf{enlace} , que puede ser fisico o simbolico. La diferencia es que mientras que uno apunta al mismo archivo el otro crea una copia de los contenidos en uno nuevo.
	\item En UNIX se utilizan los \textbf{i-nodos} para saber cuantos enlaces fisicos tiene un mismo archivo. Esto provoca que haya que controlar los bucles al recorrer todo el arbol e introduce problemas al borrarlos (se solventa disminuyendo el contador, y hasta que no sea 0, no se borra del todo)
	\item Cuando estamos trabajando desde algun directorio y nos queremos mover o referir a otro directorio o archivo, el sistema operativo ha de tener en cuenta que las \textbf{referencias} que hagamos pueden ser \textbf{absolutas (desde la raiz)} o \textbf{relativas}. Ver ls -a . y ..
\end{itemize}

\subsection{Servicios de archivos y directorios}

Cuando abrimos un archivo, lo que obtenemos es un \textbf{descriptor de archivo} que en UNIX se conoce como \textbf{i-nodo}. Por ejemplo, para la salida estandar hay unos descritores fijos: 0-in, 1-out, 2-err. Los inodos contienen todos los atributos que nos interesaran para nuestro estudio.\par
Permisos de acceso: bits: setuid(ejecutable-user), setgid(ejecutable-grupo), rwx(user), rwx(grupo), rwx(resto). \par

\section{Procesos}

\subsection{Conceptos basicos}
Un proceso es la unidad de procesamiento gestionada por el sistema operativo.\par
El sistema operativo mantiene una tabla de procesos. En esta tabla se almacena un \textbf{bloque de control de proceso} \textit{(estructura \href{https://github.com/torvalds/linux/blob/master/include/linux/sched.h}{task\_struct} en linux)} por cada proceso.
El BCP contiene la siguiente informacion:
\begin{itemize}
	\item{Identifiacion: PID, USER y relaciones padre-hijo}
	\item{Estado del procesador}
	\item{Informacion de control del proceso}
	\item{Memoria asignada}
	\item{Recursos asignados}
\end{itemize}

\subsection{threads}

Un proceso puede contener varios threads, o hilos de ejecucion que comparten algunos recursos e informacion entre ellos.\par

Cada thread se define como una funcion que se puede lanzar en paralelo con otras.\par

Los threads del mismo proceso comparten parte de su informacion: Memoria, variables globales, archivos abiertos, procesos hijos, temporizadores y semaforos. Que compartan la memoria genera que no haya proteccion de memoria y los threads son independientes unos de otros hasta cierto punto.\par
Pero no comparten: contador de programa, pila, registros y estado. En resumen, el contexto de ejecucion.\par

Las ventajas que proporciona este mecanismo de "dividir el trabajo de los procesos en tareas mas sencillas" es claro y sigue el paradigma de divide y venceras, sin embargo, cuando entramos en el terreno de la programacion concurrente hemos de ser cuidadosos por los tipicos problemas que pueden surgir.\par

\subsection{senales}

Se puede describir una senal como una interrupcion a un proceso. El comportamiento de las senales es similar al de las interrupciones en la cpu.\par

Las senales pueden provenir o bien de un proceso que la manda a otro proceso con su mismo \textit{uid} o el sistema operativo ante las tipicas excepciones que vemos cuando fallan nuestros programas (seg fault, /0...)\par

Hay muchas senales diferentes y se engloban en tres tipos fundamentales: De hardware, de comunicacion y de E/S asincrona.\par

Para que un proceso pueda enviar una senal ha de estar preparado para recibirla. Para ello tenemos que indicarle al SO el nombre de la rutina del proceso que ha de tratar ese tipo de senal. Si un proceso recibe una senal sin estar preparado para ello, se suele matar al proceso.

\subsection{Servicios de procesos}

Gestion de procesos:

\begin{itemize}
	\item{Identificador de proceso: PID (pid\_t)}
	\item{Obtener el identificador de proceso: getpid() devuelve el identificador del proceso que hizo la llamada}
	\item{Obtener el identificador del proceso padre: getppid()}
	\item{Obtener el identificador de usuario real: getuid()}
	\item{Obtener el identificador de usuario efectivo: geteuid()}
	\item{Obtener el identificador de grupo real: getgid()}
	\item{Obtener el identificador de grupo efectivo: getegid()}
	\item{Obtener el valor de una variable de entorno: *getenv( const char * name )}
	\item{Fijar el entorno de proceso: *setenv(char **env)}
	\item{Crear un proceso: fork(). Clona al proceso que lo creo, pasando a ser su hijo}
	\item{Ejecutar un programa: familia exec. Cambia el programa que esta ejecutando un proceso}
	\item{Terminar la ejecucion de un proceso: exit(int status). Similar a los return}
	\item{Esperar a la finalizacion de un proceso hijo: wait(int *status) o wait\_pid(pid\_t , int * status, int options)}
	\item{Suspender la ejecucion de un proceso: sleep(unsigned int seconds)}
\end{itemize}

Gestion de threads:

\begin{itemize}
	\item Atributos de un thread: (cada thread tiene asociado unos atributos, y los valores de estos se almacenan en un objeto de tipo \textit{pthread\_attr\_t}
		\begin{itemize}
			\item Crear atributos de un thread: int pthread\_attr\_init(pthread\_attr\_t *attr);
			\item Destruir atributos: int pthread\_attr\_destroy(pthread\_attr\_t *attr);
			\item Asignar tamano al thread: int pthread\_attr\_setstacksize(pthread\_attr\_t *attr);
			\item Obtener el tamano de la pila: int pthread\_attr\_getstacksize(pthread\_attr\_t *attr)
			\item Establecer el estado de terminacion: int pthread\_attr\_setdetachstate(pthread\_attr\_t * attr, int detachstate). El valor del segundo argumento, indicara la independencia o no del thread con respecto a otros.
			\item Obtener el estado de terminacion: int pthread\_attr\_setdetacstate(pthread\_attr\_t * attr, int * detachstate)
		\end{itemize}
	\item Creacion, identificaciony terminacion de threads:
		\begin{itemize}
			\item Crear un thread: int pthread\_create(pthread\_t *thread, pthread\_attr\_r *attr, void * (*start\_routine) (void *), void * arg). El primer argumento es el indentificador, el segundo atributos de ejecucion, el tercero el nombre de la funcion a ejecutar cuando se lance y el ultimo es un puntero a void.
			\item Obtener el identificador de un thread: pthread\_t pthread\_self(void)
			\item Esperar la terminacion de un thread: int pthread\_join(pthread thid, void **value). Es necesario indicar el thread por el que se quiere esperar.
			\item Finalizar la ejecucion de un thread: int pthread\_exit(void * value)
		\end{itemize}
\end{itemize}


\end{document}
